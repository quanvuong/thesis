\section{Discussions}

The issue of learning the wrong dependency does not surface when multi-task policies are tested in Atari tasks because their state space do not overlap \cite{ActorMimicParisotto2015,hessel2019multi,impala2018}.
Each Atari task has distinctive image-based state. The policy can perform well even when it only learns to correlate the state to the task identity. When Mujoco tasks are used to test online multi-task algorithms \cite{zintgraf2020varibad, fakoor2019meta}, the wrong dependency becomes self-correcting. If the policy infers the wrong task identity, it will collect training data which increases the overlap between the datasets of the different training tasks, correcting the issue overtime. However, in the batch setting, the policy can not collect more transitions to self-correct inaccurate task inference. Our insight also leads to exciting possibility to incorporate mechanism to quickly infer the correct causal relationship and improve sample efficiency in Multi-task RL, similar to how causal inference method has motivated new innovations in imitation learning \cite{de2019causal}. 

Our first limitation is the reliance on the generalizability of simple feedforward NN. Future research can explore more sophisticated architecture, such as Graph NN with reasoning inductive bias \cite{xu2019can,scarselli2008graph,wu2020comprehensive,zhou2018graph} or structural causal model \cite{pearl2010introduction, pearl2009causal}, to ensure accurate task inference. We also assume the learnt reward function of one task can generalize to state-action pairs from the other tasks, even when their state-action visitation frequencies do not overlap significantly. To increase the prediction accuracy, we use a reward ensemble to estimate epistemic uncertainty.
% (\autoref{sec_ensemble}). 
We note that the learnt reward functions do not need to generalize to every state-action pairs, but only enough pairs so that the task inference module is forced to consider the rewards when trained to minimize Eq. \ref{eq_triplet_task_i}. Crucially, we do not need to solve the task inference challenge while learning the reward functions and using them for relabelling, allowing us to side-step the challenge of task inference.

The second limitation is in scope. We only demonstrate our results on tasks using proprioceptive states. Even though they represent high-dimensional variables in a highly nonlinear ODE, the model does not need to tackle visual complexity. The tasks we consider also have relatively dense reward functions and not binary reward functions. These tasks, such as navigation and running, are also quite simple in the spectrum of possible tasks we want an embodied agents to perform. These limitations represent exciting directions for future work.

Another interesting future direction is to apply supervised learning self-distillation techniques \cite{xie2019selftraining, mobahi2020selfdistillation}, proven to improve generalization, to further improve the distillation procedure. To address the multi-task learning problem for long-horizon tasks, it would also be beneficial to consider skill discovery and composition from the batch data \cite{peng2019MCP, sharma2020emergent}. However, in this setting, we still need effective methods to infer the correct task identity to perform well in unseen tasks. Our explanation in Sec. \ref{sec_algo} only applies when the tasks differ in reward function. Extending our approach to task distributions with varying transition functions is trivial. Sec. \ref{sec_exp} provide experimental results for both cases.
